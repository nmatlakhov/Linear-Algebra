\documentclass[12pt]{article}
\usepackage[left=1cm, right=1cm, top=2cm,bottom=1.5cm]{geometry} 

\usepackage[parfill]{parskip}
\usepackage[utf8]{inputenc}
\usepackage[T2A]{fontenc}
\usepackage[russian]{babel}
\usepackage{enumitem}
\usepackage[normalem]{ulem}
\usepackage{amsfonts, amsmath, amsthm, amssymb, mathtools}
\usepackage{tikz}
\usepackage{tabularx}
\usepackage{hhline}

\usepackage{accents}
\usepackage{fancyhdr}
\pagestyle{fancy}
\renewcommand{\headrulewidth}{1.5pt}
\renewcommand{\footrulewidth}{1pt}

\usepackage{graphicx}
\usepackage[figurename=Рис.]{caption}
\usepackage{subcaption}
\usepackage{float}

%%Наименование папки откуда забирать изображения
\graphicspath{ {./images/} }

%%Изменение формата для ввода доказательства
\renewcommand{\proofname}{$\square$  \nopunct}
\renewcommand\qedsymbol{$\blacksquare$}

%%Изменение отступа на таблицах
\addto\captionsrussian{%
	\renewcommand{\proofname}{$\square$ \nopunct}%
}
%% Римские цифры
\newcommand{\RN}[1]{%
	\textup{\uppercase\expandafter{\romannumeral#1}}%
}

%% Для удобства записи
\newcommand{\MR}{\mathbb{R}}
\newcommand{\MQ}{\mathbb{Q}}
\newcommand{\MC}{\mathbb{C}}
\newcommand{\MK}{\mathbb{K}}
\newcommand{\MI}{\mathrm{I}}
\newcommand{\MJ}{\mathrm{J}}
\newcommand{\MH}{\mathrm{H}}
\newcommand{\MT}{\mathrm{T}}
\newcommand{\MU}{\mathcal{U}}
\newcommand{\MV}{\mathcal{V}}
\newcommand{\VN}{\varnothing}
\newcommand{\VE}{\varepsilon}
\newcommand{\id}{\mathrm{id}}
\newcommand{\Mat}{\text{Mat}}

\theoremstyle{definition}
\newtheorem{defn}{Опр:}
\newtheorem{rem}{Rm:}
\newtheorem{prop}{Утв.}
\newtheorem{exrc}{Упр.}
\newtheorem{lemma}{Лемма}
\newtheorem{theorem}{Теорема}
\newtheorem{corollary}{Следствие}

\newenvironment{cusdefn}[1]
{\renewcommand\thedefn{#1}\defn}
{\enddefn}

\DeclareRobustCommand{\divby}{%
	\mathrel{\text{\vbox{\baselineskip.65ex\lineskiplimit0pt\hbox{.}\hbox{.}\hbox{.}}}}%
}
%Короткий минус
\DeclareMathSymbol{\SMN}{\mathbin}{AMSa}{"39}
%Длинная шапка
\newcommand{\overbar}[1]{\mkern 1.5mu\overline{\mkern-1.5mu#1\mkern-1.5mu}\mkern 1.5mu}
%Функция знака
\DeclareMathOperator{\sgn}{sgn}

%Операторы ядра и образа
\DeclareMathOperator{\Ker}{Ker}
\DeclareMathOperator{\Ima}{Im}
\DeclareMathOperator{\tr}{tr}


%% Шапка для букв сверху
\newcommand{\wte}[1]{\widetilde{#1}}

%Обозначение константы
\DeclareMathOperator{\const}{\text{const}}

%Интеграл в большом формате
\DeclareMathOperator{\dint}{\displaystyle\int}
\newcommand{\ddint}[2]{\displaystyle\int\limits_{#1}^{#2}}


\newcommand{\smallerrel}[1]{\mathrel{\mathpalette\smallerrelaux{#1}}}
\newcommand{\smallerrelaux}[2]{\raisebox{.1ex}{\scalebox{.75}{$#1#2$}}}

\newcommand{\smallin}{\smallerrel{\in}}
\newcommand{\smallnotin}{\smallerrel{\notin}}

\newcommand*{\medcap}{\mathbin{\scalebox{1.25}{\ensuremath{\cap}}}}%
\newcommand*{\medcup}{\mathbin{\scalebox{1.25}{\ensuremath{\cup}}}}%

%Скалярное произведение
\DeclarePairedDelimiterX{\inner}[2]{\langle}{\rangle}{#1, #2}

%Подпись символов снизу
\newcommand{\ubar}[1]{\underaccent{\bar}{#1}}

\newcommand*\circled[1]{\tikz[baseline=(char.base)]{
		\node[shape=circle,draw,inner sep=2pt] (char) {#1};}}


\begin{document}
\lhead{Линейная алгебра}
\chead{Мануйлов В.М.}
\rhead{Лекция - 5}

\section*{Внешняя прямая сумма}
$V, \, W$ - линейные пространства над полем $\mathbb{K}$.
\begin{defn}
	\uwave{Внешней прямой суммой} $V \oplus W$ называется множество пар $(v,w) \colon v \in V, \, w \in W$ с операциями:
	\begin{itemize}
		\item[$+$ :] $(v,w) + (v^\prime,w^\prime) = (v + v^\prime, w + w^\prime)$;
		\item[$\cdot$ :] $\lambda{\cdot}(v,w) = (\lambda v, \lambda w)$, где $\lambda \in \mathbb{K}$;
	\end{itemize}
\end{defn}

Размерность внешней прямой суммы равна сумму размерностей: $\dim{(V \oplus W)} = \dim{V} + \dim{W}$.

Пусть $e_1,\dotsc, e_n$ - базис в $V$, $g_1, \dotsc, g_m$ - базис в $W$, тогда $(e_1,0),\dotsc, (e_n,0), (0,g_1),\dotsc, (0,g_m)$ - базис у внешней прямой суммы. Тогда любая пара $(v,w)$ может быть записана следующим образом:
$$
	(v,w) = (v,0) + (0,w) = \alpha_1(e_1,0) + \dotsc \alpha_n(e_n,0) + \beta_1(0,g_1) + \dotsc + \beta_m(0,g_m)
$$
\section*{Линейные отображения}
\begin{defn}
	Пусть $V,W$ - линейные пространства над $\MK$, отображение $f \colon V \to W$ называется \uwave{линейным}, если выполнено следующее:
	\begin{enumerate}[label ={(\arabic*)}]
		\item $f(a + b) = f(a) + f(b), \, \forall a,b \in V$;
		\item $f(\lambda{\cdot}a) = \lambda{\cdot}f(a), \, \forall a \in V, \, \forall \lambda \in \MK$;
	\end{enumerate}
\end{defn}

\textbf{Примеры}:
\begin{enumerate}[label ={(\arabic*)}]
	\item Нулевое отображение $f(a) = 0, \, \forall a \in V$;
	\item Изоморфизм $\Rightarrow$ линейное отображение;
	\item Система линейных уравнений $AX = B$;
\end{enumerate}

Пусть $e_1, \dotsc, e_n$ - базис в $V$, $g_1, \dotsc, g_m$ - базис в $W \Rightarrow \left\{\begin{array}{ccc}
	f(e_1) & = & a_1^1 g_1 + \dotsc + a_1^m g_m \\
	\vdots & \vdots & \vdots \\
	f(e_n) & = & a_n^1 g_1 + \dotsc + a_n^m g_m
\end{array}\right.$ - разложение по базису векторов $e_1, \dotsc, e_n \Rightarrow$ при фиксированных базисах линейное отображение задает некоторую матрицу коэффициентов.

\begin{defn}
	Матрица $A = A_f = 
	\begin{pmatrix}
		a_1^1 & \dotsc & a_n^1 \\
		\vdots & \ddots & \vdots \\
		a_1^m & \dotsc & a_n^m 
	\end{pmatrix}$ называется \uwave{матрицей линейного отображения} $f$ в базисах $e_1, \dotsc, e_n$ и $g_1, \dotsc, g_m$.
\end{defn}
Рассмотрим функцию $f(x) = f(x^1 e_1 + \dotsc + x^n e_n)$:
$$
	 f(x) = x^1f(e_1) + \dotsc + x^n f(e_n) = x^1(a_1^1 g_1 + \dotsc + a_1^m g_m) + \dotsc + x^n(a_n^1 g_1 + \dotsc + a_n^m g_m) = 
$$
$$
	= (a_1^1 x^1 + \dotsc + a_n^1 x^n) g_1 + \dotsc + (a_1^m x^1 + \dotsc + a_n^m x^n)g_m = y \Rightarrow y = Ax
$$
где $(a_1^i x^1 + \dotsc + a_n^i x^n), \, i = \overline{1,m}$ - координаты $y$. Перепишем это же выражение в матричном виде:
$$
	\begin{pmatrix}
		y^1 \\
		\vdots \\
		y^m
	\end{pmatrix} = 
	\begin{pmatrix}
		a_1^1 & \dotsc & a_n^1 \\
		\vdots & \ddots & \vdots \\
		a_1^m & \dotsc & a_n^m 
	\end{pmatrix}{\cdot}
	\begin{pmatrix}
		x^1 \\
		\vdots \\ 
		x^n
	\end{pmatrix} \Leftrightarrow y = Ax = f(x), \, y^i = a_j^ix^j = \displaystyle \sum\limits_{j = 1}^{n} a_j^i x^j
$$
Отметим следующее наблюдение: $f,h$ - линейные отображения $V \to W$, тогда определим:
\begin{enumerate}[label ={(\arabic*)}]
	\item $(f + h)(x) \coloneqq f(x) + h(x)$ - линейное отображение;
	\item $(\lambda{\cdot}f)(x) \coloneqq \lambda{\cdot}f(x)$ - линейное отображение;
\end{enumerate}
Множество всех линейных отображений из $V$ в $W$ обозначим $L(V,W)$.

\begin{lemma}
	Сопоставление линейному отображению его матрицы задает изоморфизм: 
	$$
		L(V,W) \simeq \text{Mat}_{n \times m}(\MK)
	$$
\end{lemma}
\begin{proof}
	Выбираем базисы $e_1, \dotsc, e_n \in V$ и $g_1, \dotsc, g_m \in W \Rightarrow$ отображение $F \colon L(V,W) \to \text{Mat}_{n \times m}(\MK)$ определено следующим образом:
	$$
		F(f) = A_f = 	
		\begin{pmatrix}
			a_1^1 & \dotsc & a_n^1 \\
			\vdots & \ddots & \vdots \\
			a_1^m & \dotsc & a_n^m 
		\end{pmatrix}
	$$
	Проверим, что оно является изоморфизмом:
	\begin{enumerate}[label ={(\arabic*)}]
		\item $F$ сохраняет структуру линейного пространства: $F(f_1 + f_2) = F(f_1) + F(f_2), \, F(\lambda{\cdot}f) = \lambda{\cdot}F(f)$;
		\item $F$ это биекция:
		\begin{enumerate}[label ={\arabic*)}]
			\item \uline{Инъективность}: Пусть $F(f_1) = F(f_2), \, x \in V, \, x = x^1e_1 + \dotsc + x^n e_n$, тогда:
			$$
				\forall x \in V, \, f_1(x) = (a_1^1 x^1 + \dotsc + a_n^1 x^n) g_1 + \dotsc + (a_1^m x^1 + \dotsc + a_n^m x^n)g_m = f_2(x) \Rightarrow \forall x \in V, \, f_2 = f_1
			$$
			\item \uline{Сюръективность}: Пусть $
			\begin{pmatrix}
				a_1^1 & \dotsc & a_n^1 \\
				\vdots & \ddots & \vdots \\
				a_1^m & \dotsc & a_n^m 
			\end{pmatrix}$ - произвольная $(n\times m)$ матрица, определим $f(x)$ следующим образом:
			$$
				f(x) = (a_1^1 x^1 + \dotsc + a_n^1 x^n) g_1 + \dotsc + (a_1^m x^1 + \dotsc + a_n^m x^n)g_m
			$$
			$f$ - линейное отображение, левая часть определена через правую (то есть через коэффициенты матрицы) $\Rightarrow$ сюръекция;
		\end{enumerate}
	\end{enumerate}
\end{proof}
\begin{corollary}
	$\dim L(V,W) = n{\cdot}m$.
\end{corollary}
\subsection*{Матрицы перехода к новому базису}
Пусть есть $2$ базиса $e_1, \dotsc, e_n, \wte{e}_1, \dotsc, \wte{e}_n \in V$ и 2 базиса $g_1, \dotsc, g_m , \wte{g}_1,\dotsc, \wte{g}_m \in W$. $C_v, C_w$ - матрицы перехода. 
$x \in V, \, x = (x^1, \dotsc, x^n)^T, \, \wte{x} = (\wte{x}^1,\dotsc, \wte{x}^n)^T$ - координаты $x$ в базисах $e_1, \dotsc, e_n$ и $\wte{e}_1,\dotsc, \wte{e}_n$ соответственно.
Тогда $x = C_V \wte{x}$, аналогично $y \in W, \, y = C_W \wte{y}$.

Пусть $A$ - матрица отображения $f$ в базисах $e_1,\dotsc, e_n, g_1,\dotsc, g_m$. Пусть $\wte{A}$ - матрица отображения $f$ в базисах $\wte{e}_1,\dotsc, \wte{e}_n, \wte{g}_1,\dotsc, \wte{g}_m$. Тогда: 
$$
	y =f(x) \Leftrightarrow y = A{\cdot}x \wedge \wte{y} = \wte{A}{\cdot}\wte{x}  \Rightarrow C_W\wte{y} = A{\cdot}C_V\wte{x} \Rightarrow \wte{y} = C_W^{-1}AC_V {\cdot} \wte{x}
$$
где последнее справедливо поскольку матрицы перехода - обратимы. Матрица по линейному отображению определяется однозначно при фиксированном базисе $\Rightarrow \wte{A} = C_W^{-1}AC_V$ - общий случай.

\subsection*{Образ и ядро линейных отображений}
Пусть $f \colon V \to W$ - линейное отображение.
\begin{defn}
	\uwave{Ядром отображения} $f$ называется подмножество $V$: 
	$$\Ker f = \{v \in V \mid f(v) = 0\} \subset V$$
\end{defn}
\begin{defn}
	\uwave{Образом отображения} $f$ называется подмножество $W$: 
	$$\Ima f = \{w \in W \mid \exists \, v \in V \colon f(v) = w \} \subset W$$
\end{defn}
\begin{lemma}
	$\Ker f$ и $\Ima f$ это линейные подпространства.
\end{lemma}
\begin{proof}\hfill
	\begin{enumerate}[label ={\arabic*)}]
		\item $\Ker f = \{v \in V \mid f(v) = 0\}$:
		\begin{enumerate}[label ={(\arabic*)}]
			\item $\forall v_1, v_2 \in \Ker f, \, f(v_1 + v_2) = f(v_1) + f(v_2) = 0 + 0 = 0 \Rightarrow v_1 + v_2 \in \Ker f$;
			\item $\forall \lambda \in \MK, \, \forall v \in \Ker f, \, f(\lambda {\cdot} v) = \lambda{\cdot}f(v) = \lambda{\cdot}0 = 0 \Rightarrow \lambda {\cdot} v \in \Ker f$;
		\end{enumerate}
		Таким образом $\Ker f$ - линейное подпространство $V$;
		\item $\Ima f = \{w \in W \mid \exists \, v \in V \colon f(v) = w \}$:
		\begin{enumerate}[label ={(\arabic*)}]
			\item $\forall w_1, w_2 \in \Ima f, \, \exists \, v_1, v_2 \in V \colon f(v_1) = w_1, \, f(v_2) = w_2 \Rightarrow f(v_1 + v_2) = w_1 + w_2 \in \Ima f$;
			\item $\forall \lambda \in \MK, \, \forall w \in \Ima f, \, \exists \, v \in V \colon f(v) = w \Rightarrow \lambda{\cdot}w = \lambda{\cdot}f(v) = f(\lambda {\cdot} v) \Rightarrow \lambda{\cdot} w \in \Ima f$;
		\end{enumerate}
		Таким образом $\Ima f$ - линейное подпространство $W$;
	\end{enumerate}
\end{proof}
\begin{theorem}
	$\dim{(\Ker{f})} + \dim{(\Ima{f})} = \dim{V}$
\end{theorem}
\begin{proof}
	Пусть $e_1, \dotsc, e_r$ - базис в $\Ker{f} \Rightarrow \dim{(\Ker{f})} = r$, дополним этот набор до базиса $V$, следовательно получим $e_1, \dotsc, e_r, e_{r+1}, \dotsc, e_n$ - базис в $V \Rightarrow \dim{(V)} = n \Rightarrow$ нужно показать, что $\dim{(\Ima{f})} = n - r$.
	$$
		0 = f(e_1), \dotsc, 0 = f(e_r), w_1 = f(e_{r+1}), \dotsc, w_{n-r} = f(e_n)
	$$
	Проверим, что $w_1, \dotsc, w_{n-r}$ это базис $\Ima{f}$:
	\begin{enumerate}[label ={(\arabic*)}]
		\item \uwave{Линейная независимость}: 
		$$
			\lambda_1 w_1 + \dotsc + \lambda_{n-r} w_{n-r} = 0 \Leftrightarrow \lambda_1 f(e_{r+1}) + \dotsc + \lambda_{n-r}f(e_n) = 0 \Leftrightarrow f(\lambda_1 e_{r+1} + \dotsc + \lambda_{n-r} e_n) = 0 \Leftrightarrow
		$$
		$$
			\Leftrightarrow \lambda_1 e_{n-r} + \dotsc + \lambda_{n-r} e_n \in \Ker{f} \Rightarrow \lambda_1 e_{n-r} + \dotsc + \lambda_{n-r} e_n = \alpha_1 e_1 + \dotsc + \alpha_r e_r \Rightarrow
		$$
		$$
			\Rightarrow (-\alpha_1)e_1 + \dotsc + (-\alpha_r) e_r + \lambda_1 e_{n-r} + \dotsc + \lambda_{n-r}e_n = 0 \Leftrightarrow \alpha_1 = \dotsc = \alpha_r = \lambda_1 = \dotsc = \lambda_{n-r} = 0
		$$
		поскольку $e_1, \dotsc, e_n$ это базис в $V \Rightarrow \lambda_1 = \dotsc = \lambda_{n-r} = 0 \Rightarrow e_{r+1}, \dotsc, e_n$ - линейно независимы;
		\item \uwave{Максимальность}: 
		$$
			w \in \Ima{f} \Rightarrow \exists \, v \in V \colon f(v) = w, \, v = v^1e_1 + \dotsc + v^r e_r + v^{r+1}e_{r+1} + \dotsc + v^n e_n \Rightarrow
		$$
		$$
			\Rightarrow f(v) = 0 + \dotsc + 0 + v^{r+1} f(e_{r+1}) + \dotsc + v^n f(e_n) = v^{r+1} w_1 + \dotsc + v^n w_{n-r} = w, \, \forall w \in \Ima{f}
		$$
	\end{enumerate}
	Таким образом, $w_1, \dotsc, w_{n-r}$ это базис $\Rightarrow \dim{(\Ker{f})} + \dim{(\Ima f)} = r + n -r = n = \dim{(V)}$.
\end{proof}
\subsection*{Композиция}

\begin{defn}
	Пусть $f \in L(V,W), \, h \in L(W,U), \, V \xrightarrow[]{f}W \xrightarrow[]{h}U$. Определим композицию следующим образом: 
	$$
		(h \circ f)(v)\coloneqq h\left(f(v)\right), \, h \circ f \in L(V,U)
	$$	
\end{defn}
\begin{exrc}
	Какая матрица будет у $h \circ f$: $A_h {\cdot} A_f$ или $A_f {\cdot} A_h$?
\end{exrc}
\newpage
\section*{Евклидовы и Эримтовы пространства}
Пусть $V$ - линейное пространство над $\MR$.
\begin{defn}
	\uwave{Скалярным произведением} назовем отображение $V\times V \to \MR$, если выполнено:
	\begin{enumerate}[label ={(\arabic*)}]
		\item $\forall a,b,c \in V, \, \forall \lambda \in \MR, \, (a,b + \lambda c) = (a,b) + \lambda{\cdot}(a,c)$ (линейность по второму аргументу);
		\item $\forall a,b \in V, \, (a,b) = (b,a)$ (симметричность);
		\item $\forall a \in V, \, (a,a) \geq 0$ и если $(a,a) = 0$, то $a = 0$ (положительная определенность);
	\end{enumerate}
\end{defn}
\begin{rem}
	Из свойств определения $(1)$ и $(2)$ получим линейность по первому аргументу.
\end{rem}

\begin{defn}
	Линейное пространство, на котором задано скалярное произведение называется \uwave{Евклидовым}.
\end{defn}
Пусть $V$ - линейное пространство над $\MC$.
\begin{defn}
	Отображение $V \times V \to \MC$ называется \uwave{скалярным произведением}, если выполнено:
	\begin{enumerate}[label ={(\arabic*)}]
		\item $\forall a,b,c \in V, \, \forall \lambda \in \MC, \, (a,b + \lambda c) = (a,b) + \lambda{\cdot}(a,c)$;
		\item $\forall a,b \in V, \, (a,b) = \overline{(b,a)}$;
		\item $\forall a \in V, \, (a,a) \geq 0, \, (a,a) \in \MR$ и если $(a,a) = 0$, то $a = 0$;
	\end{enumerate}
\end{defn}
\begin{rem}
	Здесь уже из $(1)$ и $(2)$ не следует линейность по $1$-му аргументу. Свойство получается немного другим:
	$$
		(b + \lambda c,a) = \overline{(a, b+ \lambda c)} = \overline{(a,b)} + \overline{\lambda}{\cdot}\overline{(a,c)} = (b,a) + \overline{\lambda}{\cdot}(c,a)
	$$
\end{rem}
\begin{defn}
	Линейное пространство над $\MC$ на котором задано скалярное произведение называется \uwave{Эрмитовым}.
\end{defn}

\section*{Примеры скалярных пространств}
$1)$ $V = \MR_{[x]}$ - пространство многочленов от $x$; $f,g \in \MR_{[x]}, \, (f,g) = \ddint{a}{b}f(x)g(x)dx$. 
\begin{proof}
	Проверим свойства скалярного произведения над $\MR$:
	\begin{enumerate}[label ={(\arabic*)}]
		\item Очевидно по линейности интеграла и подинтегральной функции; 
		\item Очевидно;
		\item $g = f \Rightarrow \ddint{a}{b}(f(x))^2dx \geq 0$;
	\end{enumerate}
\end{proof}

$2)$ $V = \Mat_{n \times m}(\MR)$; $A,\, B \in \Mat_{n \times m}(\MR), \, (A,B) = \tr(A^TB)$.
\begin{proof}
	Проверим свойства скалярного произведения над $\MR$:
	\begin{enumerate}[label ={(\arabic*)}]
		\item $\tr(A^T(B + \lambda C)) = \tr(A^TB) + \lambda\tr(A^TC)$;
		\item $\tr(A^TB) = \tr((A^TB)^T) = \tr(B^TA)$;
		\item $\tr(A^TA) \geq 0$?
		$$
			A = 
			\begin{pmatrix}
				a_1^1 & \dotsc & a_n^1\\
				\vdots & \ddots & \vdots \\
				a_1^m & \dotsc & a_n^m
			\end{pmatrix}, \, 
			A^T = 
			\begin{pmatrix}
				a_1^1 & \dotsc & a_1^m\\
				\vdots & \ddots & \vdots \\
				a_n^1 & \dotsc & a_n^m
			\end{pmatrix} 
			A^TA = 
			\begin{pmatrix}
				(a_1^1)^2 + \dotsc + (a_1^m)^2 &  \dotsc & * \\
				\vdots & \ddots & \vdots \\
				* & \dotsc & (a_n^1)^2 + \dotsc + (a_n^m)^2 
			\end{pmatrix} \Rightarrow
		$$
		$$
			\Rightarrow \tr{(A^TA)} = (a_1^1)^2 + \dotsc + (a_1^m)^2 + \dotsc + (a_n^1)^2 + \dotsc + (a_n^m)^2 \geq 0
		$$
		В этой сумме все квадраты $\Rightarrow$ все элементы $\geq 0$, а также здесь все элементы матрицы $A$. Таким образом, $\tr{(A^TA)} = 0 \Leftrightarrow \forall i,j, \, a_i^j = 0 \Leftrightarrow A =0$;
	\end{enumerate}
\end{proof} 

\end{document}