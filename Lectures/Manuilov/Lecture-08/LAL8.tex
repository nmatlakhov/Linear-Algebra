\documentclass[12pt]{article}
\usepackage[left=1cm, right=1cm, top=2cm,bottom=1.5cm]{geometry} 

\usepackage[parfill]{parskip}
\usepackage[utf8]{inputenc}
\usepackage[T2A]{fontenc}
\usepackage[russian]{babel}
\usepackage{enumitem}
\usepackage[normalem]{ulem}
\usepackage{amsfonts, amsmath, amsthm, amssymb, mathtools}
\usepackage{blkarray}
\usepackage{tikz}
\usepackage{tabularx}
\usepackage{hhline}

\usepackage{accents}
\usepackage{fancyhdr}
\pagestyle{fancy}
\renewcommand{\headrulewidth}{1.5pt}
\renewcommand{\footrulewidth}{1pt}

\usepackage{graphicx}
\usepackage[figurename=Рис.]{caption}
\usepackage{subcaption}
\usepackage{float}

%%Наименование папки откуда забирать изображения
\graphicspath{ {./images/} }

%%Изменение формата для ввода доказательства
\renewcommand{\proofname}{$\square$  \nopunct}
\renewcommand\qedsymbol{$\blacksquare$}

%%Изменение отступа на таблицах
\addto\captionsrussian{%
	\renewcommand{\proofname}{$\square$ \nopunct}%
}
%% Римские цифры
\newcommand{\RN}[1]{%
	\textup{\uppercase\expandafter{\romannumeral#1}}%
}

%% Для удобства записи
\newcommand{\MR}{\mathbb{R}}
\newcommand{\MQ}{\mathbb{Q}}
\newcommand{\MC}{\mathbb{C}}
\newcommand{\MK}{\mathbb{K}}
\newcommand{\MI}{\mathrm{I}}
\newcommand{\MJ}{\mathrm{J}}
\newcommand{\MH}{\mathrm{H}}
\newcommand{\MT}{\mathrm{T}}
\newcommand{\MU}{\mathcal{U}}
\newcommand{\MV}{\mathcal{V}}
\newcommand{\VN}{\varnothing}
\newcommand{\VE}{\varepsilon}
\newcommand{\id}{\mathrm{id}}
\newcommand{\Mat}{\text{Mat}}
\newcommand{\RE}{\operatorname{Re}}
\newcommand{\IM}{\operatorname{Im}}

\theoremstyle{definition}
\newtheorem{defn}{Опр:}
\newtheorem{rem}{Rm:}
\newtheorem{prop}{Утв.}
\newtheorem{exrc}{Упр.}
\newtheorem{lemma}{Лемма}
\newtheorem{theorem}{Теорема}
\newtheorem{corollary}{Следствие}

\newenvironment{cusdefn}[1]
{\renewcommand\thedefn{#1}\defn}
{\enddefn}

\DeclareRobustCommand{\divby}{%
	\mathrel{\text{\vbox{\baselineskip.65ex\lineskiplimit0pt\hbox{.}\hbox{.}\hbox{.}}}}%
}
%Короткий минус
\DeclareMathSymbol{\SMN}{\mathbin}{AMSa}{"39}
%Длинная шапка
\newcommand{\overbar}[1]{\mkern 1.5mu\overline{\mkern-1.5mu#1\mkern-1.5mu}\mkern 1.5mu}
%Функция знака
\DeclareMathOperator{\sgn}{sgn}

%Операторы ядра и образа
\DeclareMathOperator{\Ker}{Ker}
\DeclareMathOperator{\Ima}{Im}
\DeclareMathOperator{\tr}{tr}
\DeclareMathOperator{\rk}{rk}


%% Шапка для букв сверху
\newcommand{\wte}[1]{\widetilde{#1}}

%Обозначение константы
\DeclareMathOperator{\const}{\text{const}}

%Интеграл в большом формате
\DeclareMathOperator{\dint}{\displaystyle\int}
\newcommand{\ddint}[2]{\displaystyle\int\limits_{#1}^{#2}}


\newcommand{\smallerrel}[1]{\mathrel{\mathpalette\smallerrelaux{#1}}}
\newcommand{\smallerrelaux}[2]{\raisebox{.1ex}{\scalebox{.75}{$#1#2$}}}

\newcommand{\smallin}{\smallerrel{\in}}
\newcommand{\smallnotin}{\smallerrel{\notin}}

\newcommand*{\medcap}{\mathbin{\scalebox{1.25}{\ensuremath{\cap}}}}%
\newcommand*{\medcup}{\mathbin{\scalebox{1.25}{\ensuremath{\cup}}}}%

%Скалярное произведение
\DeclarePairedDelimiterX{\inner}[2]{\langle}{\rangle}{#1, #2}

%Подпись символов снизу
\newcommand{\ubar}[1]{\underaccent{\bar}{#1}}

\newcommand*\circled[1]{\tikz[baseline=(char.base)]{
		\node[shape=circle,draw,inner sep=2pt] (char) {#1};}}


\begin{document}
\lhead{Линейная алгебра}
\chead{Мануйлов В.М.}
\rhead{Лекция - 8}
\section*{Билинейные функции}
\begin{defn}
	Пусть $F \colon V \times V \to \MK$, где $V$ это векторное пространство над полем $\MK$, тогда $F$ это \uwave{билинейная функция}, если:
	\begin{enumerate}[label ={(\arabic*)}]
		\item $\alpha F(x_1, y) + \beta F(x_2, y) = F(\alpha x_1 + \beta x_2, y), \, \forall x_1, x_2, y \in V, \, \forall \alpha, \beta \in \MK$;
		\item $F(x, \wte{\alpha} y_1 + \wte{\beta} y_2) = \wte{\alpha} F(x,y_1) + \wte{\beta} F(x,y_2), \, \forall x, y_1, y_2 \in V, \, \forall \wte{\alpha}, \wte{\beta} \in \MK$;
	\end{enumerate}
\end{defn}

\textbf{Примеры билинейных функций}

$(1)$ Евклидово скалярное произведение: $Ve_1,\dotsc, e_n, \, F(x,y) = x_1 y_1 + \dotsc + x_n y_n$;

$\hphantom{(1)}(a)$ $F(x,y) = x_1 y_1$ - тоже билинейная функция;\\
$\hphantom{(1)}(b)$ $F(x,y) = \pm x_1 y_1 \pm x_2 y_2 \pm \dotsc \pm x_k y_k, \, k \leq n$ - тоже билинейная функция;

$(2)$ $l(x), q(x) \in V^\prime$ - двойственное пространство: $F_{l,q}(x,y) \coloneqq l(x){\cdot}q(y)$;

$(3)$ $V = C_{[a,b]}$ - пространство непрерывных функций на $[a,b]$, $f,g \in C_{[a,b]}$: $F(f,g) = \ddint{a}{b}f(x)g(x)dx$;

Далее, пусть $V$ - конечномерное пространство, где $e_1, \dotsc, e_n$ - фиксированный базис.

\begin{defn}
	\uwave{Матрицу билинейной функции} в базисе $e_1, \dotsc, e_n$ обозначим $B_F = (b_{ij})$, где $b_{ij} \coloneqq F(e_i, e_j)$.
\end{defn}

\textbf{Примеры матриц билинейных функций}

$(1)$ Скалярное произведение $\Rightarrow B_F$ - матрица Грама;

$(2)$ $F(x,y) = x_1 y_1$, матрица $F$? Подставялем $e_1, \dotsc, e_n$:
$$
	x = (x_1,\dotsc, x_n), \, y = (y_1, \dotsc, y_n) \Rightarrow B(e_i, e_j) = \begin{cases}
		1, & i = j \\
		0, & i \neq j
	\end{cases} \Rightarrow
	B_F = 
	\begin{pmatrix}
		1 & 0 & \dotsc & 0 \\
		0 & 0 & \dotsc & 0 \\
		\vdots & \vdots & \ddots & 0 \\
		0 & 0 & \dotsc & 0
	\end{pmatrix}
$$
Пусть $F(x,y)$ - билинейная функция, $x = \displaystyle \sum\limits_{i = 1}^n x_i e_i, \, y = \displaystyle \sum\limits_{j = 1}^n y_j e_j$, тогда:

$$
	F(x,y) = \displaystyle \sum\limits_{i=1}^n \sum\limits_{j = 1}^n x_i y_j F(e_i, e_j) = \sum\limits_{i=1}^n \sum\limits_{j = 1}^n x_i y_j b_{ij}
$$
То есть значение билинейной функции выражается через координаты векторов и матрицу $B$. Используем еще одну форму записи:
$$
	X = 
	\begin{pmatrix}
		x_1 \\
		\vdots \\
		x_n
	\end{pmatrix}, \,
	Y = 
	\begin{pmatrix}
		y_1 \\
		\vdots \\
		y_n
	\end{pmatrix}, \, 
	X^T = 
	\begin{pmatrix}
		x_1 & \dotsc & x_n
	\end{pmatrix} \Rightarrow
	X^T{\cdot}
	\begin{pmatrix}
		b_{11} & \dotsc & b_{1n} \\
		\vdots & \ddots & \vdots \\
		b_{n1} & \dotsc & b_{nn}
	\end{pmatrix}{\cdot}
	Y= \sum\limits_{i=1}^n \sum\limits_{j = 1}^n x_i y_j b_{ij} \Rightarrow
$$
$$
	\Rightarrow F(x,y) = \sum\limits_{i=1}^n \sum\limits_{j = 1}^n x_i y_j b_{ij} = X^T B Y
$$
Данный вид это \uwave{координатная форма записи билинейной функции}.

Априори единственность такой записи не очевидна. Сформулируем утверждение.
\begin{prop}
	Пусть $F(x,y)$ - билинейная функция, $F(x,y) = X^T B Y = X^T \wte{B} Y$, тогда $\wte{B} = B$.
\end{prop}
\begin{proof}
	$$
		\forall X,Y \colon X^T BY = X^T \wte{B} Y  \Rightarrow X^T B Y - X^T \wte{B} Y = 0 \Leftrightarrow X^T(B - \wte{B}) Y = 0, \, \forall X,Y
	$$ 
	Пусть $B - \wte{B} \neq 0$, тогда:
	$$
		\exists \, i_0, j_0 \colon \overline{b}_{i_0j_0} = b_{i_0 j_0} - \wte{b}_{i_0 j_0} \neq 0 
	$$
	Возьмем $X_0^T$ и $Y_0^T$ такие, что:
	$$
		X_0^T = 
			(0 \;\; \dotsc \;\; 0 \;\; \underset{i_0}{1} \;\; 0 \;\; \dotsc \;\;  0), \, Y_0^T = (0 \;\; \dotsc \;\; 0 \;\; \underset{j_0}{1} \;\; 0 \;\; \dotsc \;\;  0) \Rightarrow X_0^T (B - \wte{B})Y_0 = \overline{b}_{i_0 j_0} \neq 0
	$$
	Таким образом получаем противоречие $\Rightarrow B - \wte{B} = 0 \Rightarrow B = \wte{B}$.
\end{proof}
\section*{Зависимость матрицы билинейной функции от базиса}
Пусть $e_1, \dotsc, e_n$ и $e_1^\prime, \dotsc, e_n^\prime$ - два базиса в $V$, $C$ - матрица перехода/замены координат, такая, что формула замены координат имеет следующий вид:
$$
	X = CX^\prime \Leftrightarrow 
	\begin{pmatrix}
		x_1 \\
		\vdots \\
		x_n
	\end{pmatrix} = 
	\begin{pmatrix}
		c_{11} & \dotsc & c_{1n}\\
		\vdots & \ddots & \vdots \\
		c_{n1} & \dotsc & c_{nn}
	\end{pmatrix}{\cdot}
	\begin{pmatrix}
		x_1^\prime\\
		\vdots \\
		x_n^\prime
	\end{pmatrix}
$$
Тогда подставляя замену координат в билинейную функцию получим:
$$
	F(x,y) = X^T B Y = (CX^\prime)^TBCY^\prime = (X^\prime)^T (C^T B C)Y^\prime 
$$
Получим, что $B^\prime = C^T B C$ в силу однозначности координатной записи билинейной функции $\Rightarrow$ получаем, что $B^\prime$ - матрица билинейной функции  в новом базисе, $C$ - матрица перехода к новому базису.

Также возможен другой вывод данного утверждения (где нижний индекс у $c_i^k$ - это столбец):
$$
	b_{ij}^\prime = F(e_i^\prime, e_j^\prime) = F\left(\sum\limits_{k = 1}^n c_i^k e_k, \sum\limits_{s = 1}^n c_j^s e_s\right) = \sum\limits_{k = 1 }^n \sum\limits_{s = 1}^n c_i^k c_j^s F(e_k, e_s) = \sum\limits_{k = 1 }^n \sum\limits_{s = 1}^n c_i^k c_j^s b_{ks} \Rightarrow C^T B C = B^\prime
$$

Поскольку $C$ - невырожденная матрица ($\det{C} \neq 0$), то $\rk{B^\prime} = \rk{C^T B C} = \rk{B}$.
\begin{defn}
	\uwave{Ранг билинейной функции} по определению равен рангу её матрицы в каком-то базисе.
\end{defn}
Например, $\begin{pmatrix}
	1 & \dotsc & 0 \\
	\vdots & \ddots & \vdots \\
	0 & \dotsc & 1
\end{pmatrix}$ - матрица Грама в ортонормированном репере $\Rightarrow \rk{B} = n$.

Матрица вида $\begin{pmatrix}
	1 & 0 & \dotsc & 0 \\
	0 & 0 & \dotsc & 0\\
	\vdots & \vdots & \ddots & \vdots \\
	0  & 0 & \dotsc & 0
\end{pmatrix} \Rightarrow \rk{B} = 1$, матрица вида $
\begin{pmatrix}
	\pm 1 & \dotsc & 0 & \dotsc & 0 \\
	\vdots & \ddots &  \vdots & \ddots &  \vdots\\
	0 & \dotsc & \pm 1 & \dotsc & 0 \\
	\vdots & \ddots &  \vdots & \ddots &  \vdots\\
	0 & \dotsc & 0 & \dotsc & 0 	
\end{pmatrix}
 \Rightarrow \rk{B} = k$, где на диагонали стоит $k$ ненулевых элементов (либо $1$, либо $-1$).
 
\begin{defn}
	\uwave{Левое ядро} билинейной функции $\Ker_L(F) = \{v \in V \mid F(v,y) = 0, \, \forall y \in V\}$.
\end{defn}

\begin{defn}
	\uwave{Правое ядро} билинейной функции $\Ker_R(F) = \{v \in V \mid F(x,v) = 0, \, \forall x \in V\}$.
\end{defn}
\begin{defn}
	\uwave{Транспонированной} билинейной функцией называется следующее: $F^T(x,y) = F(y,x)$.
\end{defn}
\begin{prop}
	$\Ker_R\left(F^T\right) = \Ker_L(F)$.
\end{prop}
\begin{proof}
	$\Ker_R\left(F^T\right) = \{v \in V \mid F^T(x,v) = 0, \, \forall x \in V\} = \{v \in V \mid F(v,x) = 0, \, \forall x \in V\} = \Ker_L(F)$.
\end{proof}

\begin{lemma}
	$\Ker_R(F), \, \Ker_L(F)$ - подпространства в $V$.
\end{lemma}
\begin{proof}
	Рассмотрим случай левого ядра билинейной функции:
	\begin{enumerate}[label ={(\arabic*)}]
		\item $\forall v,w \in \Ker_L(F) \Rightarrow F(v,y) = F(w,y) = 0, \, \forall y \in V \Rightarrow F(v + w, y) = 0, \, \forall y  \in V$;
		\item $\forall v \in \Ker_L(F), \, \forall \lambda \in \MK \Rightarrow \lambda {\cdot} F(v,y) = F(\lambda v, y) = 0, \, \forall y \in V$;
	\end{enumerate}
	Для правого ядра билинейной функции - аналогично.
\end{proof}
\begin{lemma}
	$\dim{\left(\Ker_L(F)\right)} = \dim{\left(\Ker_R(F)\right)} = \dim{(V)} - \rk(F)$.
\end{lemma}
\begin{proof}
	Рассмотрим билинейную форму:
$$
	F(x,v) = X^TBV = \begin{pmatrix}
		x_1 & \dotsc & x_n
	\end{pmatrix}\begin{pmatrix}
	z_1 \\
	\vdots \\
	z_n
\end{pmatrix} = x_1 z_1 + \dotsc + x_n z_n = 0, \, \forall x_1, \dotsc, x_n \Leftrightarrow z_1 = \dotsc = z_n = 0
$$
тогда получим:
$$
	B\begin{pmatrix}
	v_1 & \dotsc & v_n
	\end{pmatrix}^T = B \begin{pmatrix}
	v_1 \\
	\vdots \\
	v_n
	\end{pmatrix} = \begin{pmatrix}
	0 \\
	\vdots \\
	0
	\end{pmatrix}
$$
Таким образом, $V$ - решение однородной СЛУ с матрицей $B \Rightarrow$ правое ядро совпадает с пространством решений данной системы: $BV = 0 \Rightarrow \dim{\left(\Ker_R(F)\right)} = \dim{(V)} - \rk(F)$. Отсюда получим:
$$
	\dim{\left(\Ker_L(F)\right)} = \dim{\left(\Ker_R\left(F^T\right)\right)} = \dim{(V)} - \rk\left(F^T\right) = \dim{(V)} - \rk{(F)} = \dim{\left(\Ker_R(F)\right)}
$$
\end{proof}

\end{document}