\documentclass[12pt]{article}
\usepackage[left=1cm, right=1cm, top=2cm,bottom=1.5cm]{geometry} 

\usepackage[parfill]{parskip}
\usepackage[utf8]{inputenc}
\usepackage[T2A]{fontenc}
\usepackage[russian]{babel}
\usepackage{enumitem}
\usepackage[normalem]{ulem}
\usepackage{amsfonts, amsmath, amsthm, amssymb, mathtools}
\usepackage{tikz}
\usepackage{tabularx}
\usepackage{hhline}

\usepackage{accents}
\usepackage{fancyhdr}
\pagestyle{fancy}
\renewcommand{\headrulewidth}{1.5pt}
\renewcommand{\footrulewidth}{1pt}

\usepackage{graphicx}
\usepackage[figurename=Рис.]{caption}
\usepackage{subcaption}
\usepackage{float}

%%Наименование папки откуда забирать изображения
\graphicspath{ {./images/} }

%%Изменение формата для ввода доказательства
\renewcommand{\proofname}{$\square$  \nopunct}
\renewcommand\qedsymbol{$\blacksquare$}

%%Изменение отступа на таблицах
\addto\captionsrussian{%
	\renewcommand{\proofname}{$\square$ \nopunct}%
}
%% Римские цифры
\newcommand{\RN}[1]{%
	\textup{\uppercase\expandafter{\romannumeral#1}}%
}

%% Для удобства записи
\newcommand{\MR}{\mathbb{R}}
\newcommand{\MQ}{\mathbb{Q}}
\newcommand{\MC}{\mathbb{C}}
\newcommand{\MI}{\mathrm{I}}
\newcommand{\MJ}{\mathrm{J}}
\newcommand{\MH}{\mathrm{H}}
\newcommand{\MT}{\mathrm{T}}
\newcommand{\MU}{\mathcal{U}}
\newcommand{\MV}{\mathcal{V}}
\newcommand{\VN}{\varnothing}
\newcommand{\VE}{\varepsilon}

\theoremstyle{definition}
\newtheorem{defn}{Опр:}
\newtheorem{rem}{Rm:}
\newtheorem{prop}{Утв.}
\newtheorem{exrc}{Упр.}
\newtheorem{lemma}{Лемма}
\newtheorem{theorem}{Теорема}
\newtheorem{corollary}{Следствие}

\newenvironment{cusdefn}[1]
{\renewcommand\thedefn{#1}\defn}
{\enddefn}

\DeclareRobustCommand{\divby}{%
	\mathrel{\text{\vbox{\baselineskip.65ex\lineskiplimit0pt\hbox{.}\hbox{.}\hbox{.}}}}%
}
%Короткий минус
\DeclareMathSymbol{\SMN}{\mathbin}{AMSa}{"39}
%Длинная шапка
\newcommand{\overbar}[1]{\mkern 1.5mu\overline{\mkern-1.5mu#1\mkern-1.5mu}\mkern 1.5mu}
%Функция знака
\DeclareMathOperator{\sgn}{sgn}

%Обозначение константы
\DeclareMathOperator{\const}{\text{const}}

%Интеграл в большом формате
\DeclareMathOperator{\dint}{\displaystyle\int}

\newcommand{\smallerrel}[1]{\mathrel{\mathpalette\smallerrelaux{#1}}}
\newcommand{\smallerrelaux}[2]{\raisebox{.1ex}{\scalebox{.75}{$#1#2$}}}

\newcommand{\smallin}{\smallerrel{\in}}
\newcommand{\smallnotin}{\smallerrel{\notin}}

\newcommand*{\medcap}{\mathbin{\scalebox{1.25}{\ensuremath{\cap}}}}%
\newcommand*{\medcup}{\mathbin{\scalebox{1.25}{\ensuremath{\cup}}}}%

%Скалярное произведение
\DeclarePairedDelimiterX{\inner}[2]{\langle}{\rangle}{#1, #2}

%Подпись символов снизу
\newcommand{\ubar}[1]{\underaccent{\bar}{#1}}

\newcommand*\circled[1]{\tikz[baseline=(char.base)]{
		\node[shape=circle,draw,inner sep=2pt] (char) {#1};}}
	

\begin{document}
\lhead{Линейная алгебра}
\chead{Мануйлов В.М.}
\rhead{Лекция - 2}
\section*{Линейные оболочки}
$V$ - линейное пространство над $\mathbb{K}$, $S \subset V$ - подмножество (не обязательно конечное). Если конечное: $a_1, \dotsc, a_n \in S$, то взяли бы $\lambda_1 a_1 + \dotsc + \lambda_n a_n$ - линейная комбинация.

\begin{defn}
	\uwave{Линейное комбинацией} элементов из $S$ называется конечная сумма вида $\lambda_1 a_1 + \dotsc + \lambda_n a_n$, где $\lambda_1,\dotsc,\lambda_n \in \mathbb{K}, \, a_1, \dotsc, a_n \in S$.
\end{defn}

Множество индексов $\MI \Rightarrow$ тогда можно брать сумму $\displaystyle \sum\limits_{i \smallin \MI} \lambda_i a_i$. Эта сумма определена, если все $\lambda_i$ кроме конечного числа равны $0$.

\begin{defn}
	\uwave{Линейной оболочкой}  $\langle S \rangle$ множества $S \subset V$ называется множество всех линейных комбинаций элементов из $S$.
\end{defn}

\begin{lemma}
	$\langle S \rangle$ является линейным подпространством в $V$ \bigg(доказательство через суммы вида $\displaystyle \sum\limits_{i \smallin \MI} \lambda_i a_i$\bigg).
\end{lemma}

\begin{defn}
	$S$ называется \uwave{линейно независимым}, если из равенства любой линейной комбинации элементов из $S = 0 \Rightarrow$ все коэффициенты этой линейной комбинации равны 0:
\end{defn}	
$S$ - \uline{линейно независимо}, если $\lambda_1 a_1 + \dotsc + \lambda_n a_n =0$, где $a_1, \dotsc, a_n \in S \Rightarrow \lambda_1 = \dotsc = \lambda_n = 0$.
	
$S$ - \uline{линейно зависимо}, если $\exists \, a_1, \dotsc, a_n \in S, \, \lambda_1, \dots, \lambda_n \in \mathbb{K}$, не все равные $0 \colon \lambda_1 a_1 + \dotsc + \lambda_n a_n = 0$.

Если $S \subset S^\prime$ и $S$ - линейно зависимо, то $S^\prime$ - линейно зависим.

Если $S \subset S^\prime$ и $S^\prime$ - линейно независимо, то $S$ - линейно независимо.
\begin{lemma}
	Если $\{a_1, \dotsc, a_n\}$ - линейно независимы, а $\{a_1, \dotsc, a_n, a_{n+1}\}$ - линейно зависимы, то\\ $a_{n+1}$ - есть линейная комбинация $a_1, \dotsc, a_n$.
\end{lemma}

\begin{lemma}
	Пусть $\{a_1,\dotsc,a_n\}$ и $\{b_1, \dotsc, b_m\}$ - линейно независимы и пусть $b_1, \dotsc, b_m \in \langle a_1, \dotsc, a_n \rangle$,\\ тогда $m \leq n$.
\end{lemma}
\begin{proof}
	Рассмотрим систему 
	$$
	\left\{\begin{array}{ccccccc} b_1 &=& \alpha_{11} a_1& + &\dotsc& + & \alpha_{1n}a_n \\ \vdots & & \vdots & &\ddots & &\vdots \\ b_m &=& \alpha_{m1} a_1 & + & \dotsc & + & \alpha_{mn} a_n\end{array}\right.\,
	$$ 
	$b_1, \dotsc, b_m$ - линейно независимы $\Rightarrow \lambda_1 b_1 + \dotsc + \lambda_m b_m = 0 \Rightarrow \lambda_1 = \dotsc = \lambda_m = 0 \Rightarrow$ 
	$$0 = \lambda_1 (\alpha_{11} a_1 + \dotsc + \alpha_{1n} a_n) + \dotsc + \lambda_m (\alpha_{m1} a_1 + \dotsc + \alpha_{mn}a_n) =$$ 
	$$ = (\lambda_1 \alpha_{11} + \dotsc + \lambda_m \alpha_{m1})a_1 + (\lambda_1 \alpha_{12} + \dotsc + \lambda_m \alpha_{m2})a_2 + \dotsc + (\lambda_1 \alpha_{1n} + \dotsc + \lambda_m \alpha_{mn})a_n = 0$$
	Поскольку $a_1, \dotsc , a_n$ - линейно независимы, то получим, что все коэффициенты при $a_i = 0 \Rightarrow$ получим следующую однородную СЛУ на $\lambda_1,\dotsc, \lambda_m$:
	$$
	\left\{\begin{array}{ccccccc}  \lambda_1 \alpha_{11} & + &\dotsc& + & \lambda_m \alpha_{m1} &=& 0 \\ \vdots & & \ddots & &\vdots & &\vdots \\  \lambda_1 \alpha_{1n} & + & \dotsc & + & \lambda_m \alpha_{mn} &=& 0\end{array}\right.\,
	$$ 	
	От противного: пусть $m > n$ (т.е. число неизвестных $>$ числа уравнений) $\Rightarrow \exists$ ненулевое решение $(\lambda_1, \dotsc, \lambda_m) \Rightarrow$ противоречие с $\lambda_1 = \dotsc = \lambda_m = 0 \Rightarrow m \leq n$.
\end{proof}

\section*{Размерность линейных пространств}

\textbf{Процесс определения размерности линейного пространства}
\begin{enumerate}[label ={(\arabic*)}]
	\item В $V$ есть ненулевой элемент: нет $\Rightarrow$ размерность $V = 0$, да $\Rightarrow$ есть вектор $a_1$;
	\item В $V$ есть $a_2 \colon a_1, a_2$ - линейно независимые: нет $\Rightarrow$ размерность $V = 1$, да $\Rightarrow$ есть $a_2$;
	\item В $V$ есть $a_3 \colon a_1, a_2, a_3$ - линейно независимые: нет $\Rightarrow$ размерность $V = 2$, да $\Rightarrow$ есть $a_3$;
	\item[\vdots] 
\end{enumerate}

Есть два случая:
\begin{enumerate}[label ={\arabic*)}]
	\item Процесс заканчивается на $n$-ом шаге, тогда размерность $= n$;
	\item Процесс не заканчивается, тогда  размерность $V = \infty$;
\end{enumerate}
\textbf{\uline{Обозначение}}: размерность пространства - $\dim{V}$.

Проверим независимость процесса от выбора $a_1, a_2, a_3, \dotsc$ .
\begin{proof}
	Пусть есть два выбора $a_1, a_2, \dotsc$; $b_1, b_2, \dotsc;$ если \uline{оба бесконечны}, то не важно как выбираем.
	
	\uline{Случай $1$}: $a_1,\dotsc, a_n$ - заканчивается на шаге $n$; $b_1, b_2, \dotsc$ - бесконечный. $b_1, b_2,\dotsc \in \langle a_1, \dotsc,a_n \rangle \Rightarrow$ выберем конечное $n+1 \colon b_1,\dotsc, b_{n+1} \Rightarrow$ по лемме $n+1 \leq n \Rightarrow$ противоречие.
	
	\uline{Случай $2$}: $a_1,\dotsc, a_n$; $b_1,\dotsc, b_m$ - разный набор элементов и пусть $m > n$ (и симметричная ситуация $n > m$) пусть такое реализовалось $\Rightarrow$ процесс закончится на шаге $n \Rightarrow \nexists$ элемента $a_{n+1}$, который вместе с предыдущими элементами был бы независимым $\Rightarrow \forall b \in V, \, b$ - линейная комбинация $a_1,\dotsc,a_n \Rightarrow b_1,\dotsc, b_m \in \langle a_1, \dotsc, a_n \rangle \Rightarrow$ по лемме $m \leq n \Rightarrow$ противоречие с $m > n$.
\end{proof}

\textbf{Пример}: Линейное пространство строк длины $k, \, (a_1, \dotsc, a_k)$: $k$ штук
$
	\left\{\begin{array}{c}
		(1,0, \dotsc, 0) \\
		(0,1, \dotsc, 0) \\ 
		\vdots \\
		(0, 0, \dotsc, 1)
	\end{array}	\right.
\Rightarrow \dim{V} = k$.

\textbf{Пример}: Пространство бесконечных последовательностей: $\dim{V} = \infty$.

\textbf{Пример}: Пространство многочленов степени $\leq n$. $\dim{V} = n+1\colon 1,x,x^2,\dotsc,x^n$. Если бы были линейно зависимыми, то $\exists \, \lambda_i \neq 0 \colon \lambda_0 + \lambda_1 x + \dotsc + \lambda_n x^n = 0$, но такого не может быть $\Rightarrow$ линейно независимы: берем $x = 0 \Rightarrow \lambda_0 = 0$, берем производную $\Rightarrow x = 0 \Rightarrow \lambda_1 = 0$ и так далее.

\begin{defn}
	Множество элементов в линейном пространстве $S \subset V$ назовем \uwave{максимальным}, если любой элемент $a \in V$ также принадлежит $\langle S\rangle$.
\end{defn}
\begin{lemma}
	Если $\dim{V}$ конечна, то в $V, \exists$ максимальное линейно независимое подмножество.
\end{lemma}
\begin{proof}
	По алгоритму поиска размерности линейного пространства.
\end{proof}
\begin{defn}
	Максимальное линейно независимое подмножество называется \uwave{базисом}.
\end{defn}

\textbf{Пример}: $V_1$ - пространство всех бесконечных последовательностей. $V_2$ - пространство всех финитных бесконечных последовательностей (финитная $\Leftrightarrow$ с какого-то момента начинаются только $0$).

$S = \{(1,0,0,0, \dotsc),\, (0,1,0,0,\dotsc),\, (0,0,1,0,\dotsc),\,(0,0,0,1,\dotsc), \dotsc \}$ - базис в $V_1$, но не базис в $V_2$. \\
$(1,1,1,1,\dotsc) \in V_1$ - не может быть представлена в виде конечной линейной комбинации из $S$.

Далее считаем, что все линейные пространства будут конечномерными, $\dim{V} \neq \infty$.

\begin{lemma}
	Пусть $L \subset V$ - линейное подпространство, $e_1,\dotsc, e_k$ - базис в $L$, $\dim{V} =n $. Тогда $k \leq n$ и $\exists \, e_{k+1},\dotsc, e_n$ такие, что $e_1,\dotsc,e_n$ - базис в $V$ (т.е. базис подпространства можно дополнить до базиса объемлющего пространства).	
\end{lemma}
\begin{proof}
	Если $n < k$, то $e_1,\dotsc, e_k \in L \subset V \Rightarrow$ противоречие $\Rightarrow n \geq k$.
	
	Ищем элемент с номером $k+1$: линейно независим с предыдущими $e_1,\dotsc, e_k, e_{k+1}$ - линейно независимы. Если такого нет, то $\forall a \in V \Rightarrow a \in L$. Если нашли $\Rightarrow$ ищем дальше до тех пор, пока не сможем найти следующий элемент $\Rightarrow \dotsc \Rightarrow$ дошли до размерности пространства $V$.
\end{proof}

\begin{corollary}
	Если $L \subset V$ и $\dim{L} = \dim{V}$, то $L = V$.
\end{corollary}
\begin{proof}
	(От противного): Пусть $L \neq V \Rightarrow \exists \, a \in V \colon a\notin L \Rightarrow e_1, \dotsc, e_k$ - базис в $L$. Добавим к нему $a \Rightarrow e_1, \dotsc, e_k, a$ - линейно независимы. Иначе $a \in \langle e_1,\dotsc, e_k \rangle \Rightarrow \dim{V} > \dim{L} \Rightarrow$ противоречие.
\end{proof}

\begin{prop}
	Пусть $e_1, \dotsc, e_n$ - базис в $V \Rightarrow \forall x \in V,\, \exists \, x_1, \dotsc, x_n \in \mathbb{K}\colon x = x_1 e_1 + \dotsc + x_n e_n$, причем данное выражение - единственно.
\end{prop}
\begin{proof}
	Пусть $x = x_1 e_1 + \dotsc + x_n e_n, \, x = y_1 e_1 + \dotsc + y_n e_n \Rightarrow 0 = (x_1 - y_1)e_1 + \dotsc + (x_n - y_n)e_n \Rightarrow$ так как $e_1, \dotsc, e_n$ - линейно независимы $\Rightarrow x_i - y_i = 0 \Rightarrow x_i = y_i$.
\end{proof}
\begin{defn}
	$\forall x \in V, \, e_1, \dotsc, e_n$ - базис в $V$, $\exists! \, x_1, \dotsc, x_n \in \mathbb{K} \colon x = x_1 e_1 + \dotsc + x_n e_n$, которые называются \uwave{координатами элемента $x$ в базисе} $e_1,\dotsc, e_n$.
\end{defn}

Пусть $e_1, \dotsc, e_n; \, \widetilde{e}_1, \dotsc, \widetilde{e}_n$ - базисы в $V$. Выразим $\left\{\begin{array}{ccc} \widetilde{e}_1 &=&c_{11}e_1 + \dotsc + c_{n1}e_n\\
\vdots&\vdots&\vdots \\
\widetilde{e}_n &=&c_{1n}e_1 + \dotsc + c_{nn}e_n
\end{array}	\right. \Rightarrow$ получим матрицу перехода:
$$
	\begin{pmatrix} 
		\widetilde{e}_1 & \dotsc & \widetilde{e}_n 
	\end{pmatrix} 
	= 
	\begin{pmatrix} 
		e_1 & \dotsc & e_n 
	\end{pmatrix}
	{\cdot} 
	\begin{pmatrix} 
		c_{11} & \dotsc & c_{1n}\\ 
		\vdots & \ddots & \vdots\\ 
		c_{n1} & \dotsc & c_{nn}
	 \end{pmatrix}, \, 
 	C = 
 	\begin{pmatrix} 
 		c_{11} & \dotsc & c_{1n}\\ 
 		\vdots & \ddots & \vdots \\ 
 		c_{n1} & \dotsc & c_{nn}
 	\end{pmatrix}
$$
$C$ - матрица перехода к новым координатам ($C$ - невырожденная, $|C| \neq 0$).

$x \in V \Rightarrow x = x_1 e_1 + \dotsc + x_n e_n = \widetilde{x}_1 \widetilde{e}_1 + \dotsc + \widetilde{x}_n \widetilde{e}_n = \widetilde{x}_1(c_{11} e_1 + \dotsc + c_{n1}e_n) + \dotsc + \widetilde{x}_n(c_{1n} e_1 + \dotsc + c_{nn}e_n) = $\\
$ = (\widetilde{x}_1 c_{11} + \dotsc + \widetilde{x}_n c_{1n})e_1 + \dotsc + (\widetilde{x}_1 c_{n1} + \dotsc + \widetilde{x}_n c_{nn})e_n \Rightarrow$ коэффициенты при векторах $e_1, \dotsc, e_n$ - совпадают, тогда:
$$
	\left\{
	\begin{array}{ccc} 
		x_1 &=& \widetilde{x}_1 c_{11} + \dotsc + \widetilde{x}_n c_{1n}\\
		\vdots&\vdots&\vdots \\
		x_n &=&\widetilde{x}_1 c_{n1} + \dotsc + \widetilde{x}_n c_{nn}
	\end{array}	\right. \Rightarrow
	\begin{pmatrix} 
		x_1 \\ \vdots \\ x_n
	\end{pmatrix} 
	= 
	\begin{pmatrix} 
		c_{11} & \dotsc & c_{1n}\\ 
		\vdots & \ddots & \vdots\\ 
		c_{n1} & \dotsc & c_{nn}
	\end{pmatrix}
	\begin{pmatrix} 
		\widetilde{x}_1 \\ \vdots \\ \widetilde{x}_n 
	\end{pmatrix} 
	= 
	C
	\begin{pmatrix} 
		\widetilde{x}_1 \\ \vdots \\ \widetilde{x}_n 
	\end{pmatrix} 
$$

\textbf{\uline{Обозначение}}: $x^i y_i = \displaystyle \sum\limits_{i = 1}^{n} x^i y_i = x^1y_1 + \dotsc + x^n y_n$, один индекс верхний, другой нижний $\Rightarrow$ будет подразумеваться суммирование.

\end{document}